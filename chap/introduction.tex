\specialchap{绪言}

  随着中文字的大量数字化,中文的电子咨询流通以及文化传播通过互联网得到了极大地扩展。现在的互联网上,中国网友之间来往密切,在不同设备上输入中文的情况也越来越普遍。而在中文的输入过程中,中文输入法是一个用户输入到计算机输出的不可避免中间层,也是一个重要人机交互方式。可以说,用户输入的速度,准确度等指标,直接受输入法设计影响。

  目前我国普遍存在的中文输入法种类繁多,而大陆用户则主要使用标准拼音输入为基础的拼音输入法为多。从输入界面上分,又分为QWERTY键盘拼音输入法和手机九宫格输入法。但这两种输入方式都有着单⼀特征编码冗余多,回忆式操作模式门槛⾼等等局限性。而且,在触摸设备上,由于没有键盘的按键反馈,并没有实体键盘上可以盲打输入的优势。在触摸设备飞速发展的今天,我们有必要考虑拼音输入法的下一步进化方向。

  本文详细介绍了两种在触摸设备上中文输入方法,分别面向对拼音输入生疏和熟悉的两种不同的用户群,综合利用多点触摸和多通道,意在为不同的用户群体提供更加便捷易用的交互手段。

