\specialchap{绪言}

  随着中文字的大量数字化,中文的电子咨询流通以及文化传播通过互联网得到了极大地扩展。现在的互联网上,中国网友之间来往密切,在不同设备上输入中文的情况也越来越普遍。而在中文的输入过程中,中文输入法是一个用户输入到计算机输出的不可避免中间层,也是一个重要人机交互方式。可以说,用户输入的速度,准确度等指标,直接受输入法设计影响。

  目前我国普遍存在的中文输入法种类繁多,而大陆用户则主要使用标准拼音输入为基础的拼音输入法为多。从输入界面上分,又分为QWERTY键盘拼音输入法和手机九宫格输入法。但这两种输入界面都有源自于拉丁字母键盘,用于输入拼音有着诸多局限性。而且,多点触摸设备发展飞速,其支持的不同种类的交互手段也越来越多。而目前的输入法并没有充分利用。在触摸设备覆盖率日益增长的今天,我们有必要考虑拼音输入法的下一步进化方向。

  本文提出了两种在触摸设备上的,面向不同用户群体的,新型的中文输入法的设计方案与实现效果。第一章介绍了拼音输入的背景,和现目前主流的拼音输入法,分析了他们的优劣。第二章介绍了一种新型的,面向初学者的拼音输入法的设计和实现,并讨论了其优势。第三章介绍了一种新型的,面向熟练者的拼音输入法的设计和实现,并讨论了其优势。第四章讨论了这两种设计实现目前所存在的缺陷以及能够改进的地方。
