\chapter{面向熟练者的拼音输入法设计}
  \section{设计初衷}

  大多数18到30岁左右用户都受过良好的教育,拼音基础知识牢固,已经有十年以上的拼音输入法使用经验,并且能够熟练地使用除基础输入以外更多新增特性。(见第\ref{sec:new_feature}节)在上一章的讨论中,我们把这一部分用户列为非初学者用户,不妨称其为熟练者。由于熟练者对拼音掌握的程度远远高于初学者,我们可以假定,他们在拼音形态的辨识,拼音与汉字的匹配方面不存在任何困难。因此他们在触摸设备上的输入体验更多的局限于现有键盘的布局和交互技术。

  正如之前第\ref{sec:layout_defect}节所讨论的一样,现有拼音输入法键盘并不是为拼音输入定制的。无论是QWERTY键盘还是九键键盘,都有其分别的缺陷。不仅如此,目前主流输入法还是采用模仿实体键盘的按键式的设计,将屏幕上的局域划分成一系列虚拟的按钮,从而形成一个虚拟键盘。而由于虚拟键盘并不能提供像实体键盘一样的按键回馈,常常会有误操作发生。这种误操作,虽然会随着屏幕尺寸增大而得到改善,但仍没有根本上的解决方法。本设计尝试使用没有更大尺寸的设备为更多丰富的交互提供了足够的空间。

  现有的拼音输入法大多采用点击输入的方式。无论单点触摸设备还是多点触摸设备,其基本的交互方式都是在屏幕同一位置按下和抬起,手指离开屏幕后才会有位移动作。

  \section{设计分析}
  \subsection{布局设计}

  \subsection{交互设计}

  \section{具体实现}
  \subsection{整体布局}
  \subsection{输入区布局设计}
  \subsection{输入区交互设计}

  \subsection{本设计的优势}

  同样,考虑第\ref{sec:limit}节所提到的现有输入法的局限性,本设计的优势有以下几点:

  \begin{enumerate}
  \item

  \end{enumerate}
