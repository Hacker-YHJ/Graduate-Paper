\chapter{面向熟手的拼音键盘设计}
	\section{用户分析}
	\begin{enumerate}
		\item[\textbf{Q:}]
		我的编译结果很奇怪,文字很靠近页面的顶端。请问这是怎么回事?

		\item[\textbf{A:}]
		请检查你的程序设置。
		如果使用~WinEdt,可点击~Options,选择~Execution Modes,
		检查一下~dvips、dvipdfmx、ps2pdf~等程序的纸张设置。

		\item[\textbf{Q:}]
		打印论文时不希望使用彩色的链接,请问应该怎么办?

		\item[\textbf{A:}]
		\verb|\hypersetup{colorlinks=false}|。
		关于书签和链接的问题,
		请参阅~hyperref~宏包的文档\supercite{hyperref-doc}。

		\item[\textbf{Q:}]
		导言区的内容好多,应该有好多在我的论文里是不必要的。
		请问可以去掉哪些?

		\item[\textbf{A:}]
		如果你使用~GBK~编码,则~pkuthss~文档类的~UTF-8~选项是不必要的。
		如果你不需要生成的~pdf~里的书签和链接,则~hyperref~宏包是不必要的,
		同时用于进行相关设置的~%
		\verb|\hypersetup|~和~\verb|\pdfbookmark|~命令也应该去掉。
		如果你不使用~\verb|\verbatiminput|~命令和~\verb|comment|~环境,
		则~verbatim~宏包是不需要的。
		如果你不需要上标的引用记号,则~\verb|\supercite|~宏可以去掉。
		如果你不需要使用密集的罗列环境,则~\verb|\denselist|~宏可以去掉。

		wasysym~宏包不应该去掉,
		因为~\verb|chap/originauth.tex|~中使用了其提供的~\verb|\Box|~命令。
		设置页面居中和行距的命令不建议去掉:
		如果改变这些设置,虽然不会对排版效果造成致命的影响,
		但影响可能还是很显著的。

		\item[\textbf{Q:}]
		文档里面“致谢”一章的书签链接到的位置不对,请问这是为什么?

		\item[\textbf{A:}]
		这应该是由上游的~ctex~宏包的一个问题造成的。
		在~\verb|\backmatter|~以后,
		即使用~\verb|\chapter|~命令开始的章节也不会被编号,
		但会计入目录和产生书签。
		使用当前版本的~ctexbook~文档类时,
		产生的~pdf~文档的这一类书签和链接指向的位置常常是错误的。
		这个问题应该正在修复中;在问题解决之前,
		一个缓解问题的办法是将~\verb|\backmatter|~以后%
		的~\verb|\chapter|~命令全部改为~\verb|\specialchap|~命令。
	\end{enumerate}

	\section{设计理念}

	一个问题是~\ref{sec:faq}~中提到的书签的问题。
	这个问题应该很快能够得到解决。

	此外,还应该注意到,
	研究生手册\supercite{F13}和其电子版要求的论文封面并不一致。
	这里以电子版为准。

	\section{具体实现}

	关于~pkuthss~文档模板的意见和建议,
	请到北大未名~BBS~的~MathTools~版提出。
	谢谢 :)

