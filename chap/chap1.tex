\chapter{背景介绍}
	\section{中文拼音简介}

	目前在中国大陆,一般的学前教育或者小学教育都是选汉语拼音作为中文教育的起点。汉语拼音,或又称拼音,是一种以拉丁字母作汉字标音的方案。\supercite{wjm}汉语拼音使用拉丁字母和标注在字母之上的一些附加符号来表示汉语发音。参考现代音韵学中对汉语音节结构的划分,可以将构成汉语拼音的成分分为声母、韵母和声调三部分。现代拼音输入法主要考虑声母和韵母两部分作为检索汉字的输入。

	按照汉语拼音方案《声母表》中的规定,声母由汉语中每一个音节的起始辅音构成。在中国诸多方言中,对辅音有不同程度的混合和模糊。如西南官话中的四川方言,对卷舌音和齿龈音不加区分,部分边音与鼻音不加区分;闽南语的泉漳片方言对唇齿音和软颚音音不加区分等等。\supercite{jdp}相对的,韵母主要有汉语中没一个音节的元音构成。按照韵母结构,又可将韵母分为单韵母、复韵母和鼻韵母三类。

	在汉语拼音系统中,还有语流音变现象,包括变调、轻声、儿化、音变等等。但这些现象在现代汉语拼音输入法中均没有得到使用,在此略过不谈。

	\section{现有拼音输入法}

	现代主流的拼音输入法,根据
	\begin{itemize}\denselist
		\item 一个基本的~\LaTeX{}~发行版。
		\item CJK~或~xeCJK(供~Xe\LaTeX{}~使用)宏包。
		\item ctex~宏包\supercite{ctex-doc,ctexfaq}%
			(提供了~ctexbook~文档类)。
		\item 中文字体。
		\item 如果需要使用~Makefile~来实现自动编译,还需要~Make~工具;
			但如果使用由批处理实现的伪“Makefile”就不用了。
	\end{itemize}

	最新的~\TeX{}Live~系统和~\CTeX~套装都已经包含%
	除中文字体之外所有要求的项目;中文字体需要用户自行获得。

	Linux~用户可以从软件源获得~GNU~的~make;
	其它类~UNIX~系统应该也会提供~make~工具,请参阅相应的文档以获得帮助。
	Windows~用户可以从以下地址下载~Windows~下的~GNU make~工具:

	\url{http://gnuwin32.sourceforge.net/packages/make.htm}(国际网)
	\vspace{-0.1em}\par
	\url{http://c.pku.edu.cn/software/c/mingw-c.7z}\footnote%
	{\ 感谢曹东刚老师在教学网站提供~GNU make~的下载。}(北大校园网)

	为了获得最好的支持,我们建议用户使用最新版的~\LaTeX{}~系统和各宏包。

	\section{现有输入法的局限性}

	pkuthss~文档模板支持三种编译方式,即
	\begin{itemize}\denselist
	  \item \LaTeX{} -- dvipdf~方式:
		即顺次执行~\verb|latex|,\verb|bibtex|,%
		\verb|latex|,\verb|latex|,\verb|dvipdfmx|。
	  \item pdf\LaTeX{}~方式:
		即顺次执行~\verb|pdflatex|,\verb|bibtex|,%
		\verb|pdflatex|,\verb|pdflatex|。
	  \item Xe\LaTeX{}~方式:
		即顺次执行~\verb|xelatex|,\verb|bibtex|,%
		\verb|xelatex|,\verb|xelatex|。%
		\emph
		{%
			注意:Xe\LaTeX{}~对非~UTF-8~的编码支持不好,
			因此Xe\LaTeX{}~方式的编译不支持~GBK~编码。
		}
	\end{itemize}

	pkuthss~文档模板附带的~Makefile~中已经对这三种编译方式进行了完整的配置。
	用户只需要在~Makefile~中通过设定变量~\verb|JOBNAME|~的值%
	指定被编译的主文件名,
	并通过设定变量~\verb|LATEX|~的值指定采用哪种编译方式,
	即可通过在主文件所在目录调用~Make~工具来实现自动编译:
	如果是在类~UNIX~环境下,则用户应该调用的命令名为~\verb|make|;
	而如果是在~Windows~环境下,
	则用户应该调用的命令名为~\verb|mingw32-make|。

	用户如果不想配置~Windows~下的~GNU Make,
	则也可以使用由~Windows~批处理实现的伪“Makefile”,
	通过在主文件所在目录调用~\verb|make|\footnote%
	{\ %
		Windows~将批处理文件作为可执行文件,
		调用时可以不显式地指出扩展名。
	}~或直接双击~\verb|make.bat|~的图标运行之。%
	\emph
	{%
		注意:这样不能自动生成编译所需的部分图片。
		用户可能需要进入~\texttt{img/}~目录%
		执行那里的~\texttt{make.bat}~来手动生成这些图片。
	}

