\chapter{后续拓展与讨论}

本文介绍的两个输入法设计,都还处于比较初级的阶段。虽然已经基本实现了初始的设计初衷,做出了demo程序,并通过调试可以正常使用。本文所提出的关于两种不同的输入法的设计仍然还有很多可以提升的地方。

\section{方言地区用户优化\label{sec:dialect}}

正如第\ref{sec:intro}节所说,中国存在着很多方言地区,而方言的使用者对标准拼音的掌握是有偏差的。现代拼音输入法对方言所作出的优化已在第\ref{sec:new_feature}节介绍过。而对本文介绍的两种输入法来说,需要在传统方法上加以改进,才能使用。

声母的情况较为简单。在面向熟练者的声母输入区上,相邻两个键位融合到了一起,其实是对方言地区用户优化的一项尝试。如西南官话中的四川方言,对 “l”,“n” 不加区分,就可以有选择的将这两个键合并到一起。最终想要达到的效果是,用户有选择的指定某一个自己所在的方言地区,系统自动判断并融合对应声母键位。

韵母的情况就比较难处理。如果是面向初学者的输入法,可以仿照声母的思路,融合易混淆的键位。而面向熟练者的设计,由于其拆分的特性,并不能照此思路继续。笔者还没有想出非常合适的方法应用在这种情况下。然而,退一步思考,既然是使用拼音输入法的熟练者,是否可以假设他对拼音的掌握已经克服了方言对其的影响。这是一个值得探讨的问题。

\section{词组,整句输入}

由于目前开发的程序还处于demo性质,因此没有加入词组和整句输入的功能,目前用户只能单字输入。即使输入单个汉字后,程序也不会提示高频搭配字,如“高”与“高兴”,等等。

加入词组输入和整句输入,意味着需要一份更大的词典文件,用于记录常出现的词组和其频率。这就需要输入法做出更多优化,因为一个优秀的输入法不应该过多的占有设备的内存,而让其他程序的运行变得缓慢。当然,还需要加入复杂的算法判断输入时的上下文,以辅助用户选择尽可能正确的汉字备选项,从而减轻用户的使用负担。这一方面,现有输入法已经做得很好了,可以向其参考借鉴。

\section{更紧凑的布局}

由于是在大型触摸设备上做的demo程序,并没有考虑键盘尺寸等等问题,先全部占用了屏幕空间。考虑到目前的主流输入法最多只会占用一半屏幕的空间,更有紧凑的布局使用大概1/3的屏幕空间。本设计如果真的要投入使用,势必需要调整为更加紧凑的布局。

\section{更准确快速的拟合算法}

当前轨迹输入使用的拟合算法相当简单直接,假设设备会记录用户的滑动操作信息\(N\)条,那么该程序的运行时间为\(O(N * M * K)\),其中\(M\)为总共的韵母数量,\(K\)为组成韵母的元音个数的品均值。而且,本算法并没有学习的成分,有时候的表现并不只能。

触摸设备虽然近来发展飞速,但是其性能仍不能和桌面电脑相比。使用更优质的算法,或者将机器学习的思想加入到算法中,使得轨迹匹配过程能够更加快速准确,是提升面向熟练者输入法的一大重要途径。
