\chapter{致谢}

在学士学位论文即将完成之际,我想向曾经给我帮助和支持的人们表示衷心的感谢。首先要感谢我的导师王衡教授,她在学习和科研方面给了我大量的指导,并为我们提供了良好的科研环境,让我学到了知识,掌握了科研的方法,也获得了实践锻炼的机会。她严谨的治学态度、对我的严格要求使我受益匪浅。除此之外,她在《人机交互》课程上毫无保留的解答了我对未来发展方向的解惑,也让我受用终身。在此向她表示由衷的感谢!

感谢我已经毕业的师兄邓德重、邓景文,师姐化静,他们曾经给了我无私的帮助和鼓励,让我学到很多。感谢实验室同门钱锦,王文铎,他们丰富的经验和科研的热情对我的论文设计提供了很大的支持。感谢同届的徐鹏,师弟吕鑫,江天源,他们是我学习、工作和生活上的伙伴,也是面对困难和挑战时的战友。和他们在一起的日子是本科期间快乐的时光。

感谢我在可视化实验室实习时,我的导师袁晓如对我的悉心教导。他在我对科研一无所知的时候,让我参与内部组会,熟悉计算机科学研究,尤其是可视化方向研究的流程,让我领略了科研的严谨冷峻的魅力。虽然我最后没有在可视化实验室一直实习下去,但他对我的投入,与我之后的实习的选择以及未来方向的选择,都是分不开的。

感谢在北京微软亚洲研究院实习时的同事们,他们在我第一次参加实际项目开发的过程中给了我莫大的帮助和鼓励。特别要感谢我的项目经理张海东,我的mentor Ray,胡志涛,是他们的信任给了我很多锻炼的机会,也一直对他们给予我的生活上的照顾心存感激。和他们一起为可视化项目奋战的半年多是我人生中一段难忘的经历。

感谢我在创业公司的同事们,董未、向仁凯、张哲、王杨,他们的精湛专业技能和严谨的做事风格永远是我学习的榜样。更多我无法逐一列出名字的朋友,他们给了我无数的关心和鼓励,也让我的本科校外生活充满了温暖和欢乐。我非常珍视和他们的友谊!

感谢信息科学技术学院计算机系3班的同学,感谢他们在学习和生活上给予我的帮助。

感谢我的父母,他们给了我无私的爱,我深知他们为我求学所付出的巨大牺牲和努力,而我至今仍无以为报。祝福他们,以及那些给予我关爱的长辈!

还有很多我无法一一列举姓名的师长和友人给了我指导和帮助,在此衷心的表示感谢!

最后,衷心感谢在百忙之中抽出时间审阅本论文的专家教授。
