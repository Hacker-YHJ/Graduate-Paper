\specialchap{结论}

本文在介绍了现有拼音输入法,通过调研分析了现有输入法对某一些用户的不适用性,通过辅助性程序分析了现有拼音输入法键盘的非最优性。之后,本文将现有主要拼音输入法用户根据对汉语拼音及拼音输入法的熟悉程度分为两类,详细讨论了两种不同用户群对中文拼音输入法的需求差异,分析了现有输入法对于不同用户群的不同劣势。

针对两组用户的需求差异,本文分别做出了面向这两种用户的输入法设计,研讨了设计初衷,提出了设计方案,并初步在Android设备上实现了设计。对于初学者用户群体,采用了全声母韵母分离式键盘设计,结合多通道、多点触控技术,为其创造了一个记忆负担小,上手快,有一定辅助学习功能的输入环境。对于熟练者用户群体,采用了全声母键盘,辅以精简韵母键盘,在前者的基础上,充分利用了现代多点触控设备提供的滑动交互功能,为用户提供了一种新型的,交互形式丰富,占用空间少,有一定盲打可能性的输入环境。

当然,设计本身和初步的实现都还有很多不足的地方。本文提出了一些目前实现的不足之处,对其后续发展提出了相应的讨论。如对整句输入的支持,布局的改进以及算法的优化等等。除此之外,我相信在多点触摸设备上,尤其是大型设备上的拼音输入法还有更多的地方值得探索和挖掘。
