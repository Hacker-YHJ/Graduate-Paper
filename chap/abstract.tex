% !TEX encoding = UTF-8 Unicode
% -*- coding: UTF-8; -*-

% 摘要要求在 3000 字以内。

\cleardoublepage
\begin{cabstract}

  本文详细介绍了两种在触摸设备上的新型中文拼音输入方法。两种输入方法分别面向对拼音输入生疏和熟悉的两种不同的用户群,综合利用了触摸设备所提供的多点触摸功能,采用多通道输入的方式,意在为不同的用户群体提供更加便捷易用的交互手段。同时,本文对这两种新型输入法与多种传统拼音输入法进行了比较,包括了不同设备上的,拥有不同键盘布局的输入法,并分析了该输入法在触摸设备上的优势。最后,本文讨论了该输入法现阶段的不足已经的其后续拓展的方向。

\end{cabstract}

\cleardoublepage
\begin{eabstract}

  We present two kinds of Chinese Input Methods for multitouch devices. These two methods are aimed for useres who are familiar with Pinyin, as well as those who are unfamiliar with it, by taking the advantages of interactions provided by multitouch device and using multichannel techniques. We intend to provide more convenient and easier access to Chinese input for various users. Meanwhile, We compare our input methods to those already existed ones, including those on different devices and those have different layout. And we point out the advantages of our approach. Finally, we discussed the remaining problems of our input methods for now and the possibilities of their future developments.

\end{eabstract}
